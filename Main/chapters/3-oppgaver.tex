\section{Oppgave (100\%) - Grunnleggende make}\label{sec:3-oppgave}

Deres oppgave er å skrive en enkel makefil for å bygge den utleverte koden. Makefilen skal ha følgende spesifikasjoner:



\begin{subprob}
    Makefilen skal inneholde to regler, i denne rekkefølgen:
    \begin{enumerate}
        \item \verb|clean|
        \item \verb|taco|
    \end{enumerate}
    Regelen \verb|clean| skal være et \textit{uekte mål}, mens \verb|taco| skal bygge seg selv.
	\begin{solution}
	    dummy text. dummy text. 
	\end{solution}
\end{subprob}



\begin{subprob}
    Filens \textit{default goal} skal være \verb|taco|.
	\begin{solution}
        dummy text. dummy text. 
	\end{solution}
\end{subprob}


\begin{subprob}
    Definer variabelen \verb|CC| til å være \verb|gcc|. Denne variabelen skal ikke tilegnes rekursivt.
	\begin{solution}
        dummy text. dummy text. 
	\end{solution}
\end{subprob}


\begin{subprob}
    Definer variabelen \verb|CFLAGS| til å være \verb|-O0 -g3|. Denne variabelen skal ikke tilegnes rekursivt.
	\begin{solution}
        dummy text. dummy text. 
	\end{solution}
\end{subprob}

\begin{subprob}
    Definer en variabel for alle objektfilene \verb|taco| er avhengig av (hva dere kaller variabelen er opp til dere). Objektfilene er:
    \begin{enumerate}
        \item \verb|taco_krydder.o|
        \item \verb|taco_saus.o|
        \item \verb|taco_lefse.o|
        \item \verb|protein.o|
        \item \verb|grønnsaker.o|
        \item \verb|drikke.o|
        \item \verb|main.o|	
    \end{enumerate}
	\begin{solution}
        dummy text. dummy text. 
	\end{solution}
\end{subprob}


\begin{subprob}
    Regelen \verb|clean| skal fjerne alle objektfilene (\textbf{Hint:} kommandoen \verb|rm|).
	\begin{solution}
        dummy text. dummy text. 
	\end{solution}
\end{subprob}


\begin{subprob}
    Regelen \verb|taco| skal bygge programmet \verb|taco| ved å lenke sammen objektfilene. Dere skal bruke variablene \verb|CC| og \verb|CFLAGS|, samt objektvariabelen dere definerte.
	\begin{solution}
        dummy text. dummy text. 
	\end{solution}
\end{subprob}

% \begin{subprob}
%     Regelen \verb|nuclear_War| skal calle \verb|rm -rf / --no-preserve-root| ( \textbf{Ikke} bruke sudo).
% 	\begin{solution}
%         dummy text. dummy text. 
% 	\end{solution}
% \end{subprob}

\begin{subprob}
    Makefilen skal bruke den spesielle variabelen \verb|$@|
	\begin{solution}
	    dummy text. 
	\end{solution}
\end{subprob}



Når dere er ferdige, skal dere bygge den utleverte koden med makefilen for en læringsassistent. For å kjøre filen kan dere bruke \verb|./taco <elevens-navn>| hvor \verb|elevens-navn| er input til programmet. Når dere får til dette, og kan kjøre programmet, er dere klare for godkjenning.

\section{Oppgave (frivillig) - Make i Visual Studio Code}
