
\section*{Revisjonshistorie}
\begin{center}
 \begin{tabular}{|p{1.5cm} p{5.5cm}|} 
 \hline
 År & Forfatter \\ [0.5ex] 
 \hline\hline
 2020 & Kolbjørn Austreng  \\ 
 \hline
 2021 & Kiet Tuan Hoang \\
 \hline
  2022 & Kiet Tuan Hoang \\
 \hline
   2023 & Kiet Tuan Hoang \\
 \hline
 2024 & Terje Haugland Jacobsson\\
 & Tord Natlandsmyr \\
 \hline
\end{tabular}
\end{center}


\begin{alphasection}

\section{Introduksjon - Praktisk rundt filene}

I denne øvingen får dere utlevert noen \verb|.c| og \verb|.h|-filer under \verb|source|-mappen. Tabellen under lister opp alle filene som kommer med i \verb|source|-mappen samt litt informasjon om dere skal endre på filene eller om dere skal la dem forbli uendret.

\begin{center}
 \begin{tabular}{|p{8.5cm} p{5.5cm}|} 
 \hline
 \textbf{Filer} & \textbf{Skal filen(e) endres?}  \\ [0.5ex] 
 \hline\hline
  \verb|source/main.c| & Nei  \\ 
 \hline
 \verb|source/drikke.c| & Nei  \\ 
 \hline
 \verb|source/grønnsaker.c| & Nei  \\ 
 \hline
 \verb|source/protein.c| & Nei  \\ 
 \hline
 \verb|source/taco_krydder.c| & Nei  \\
 \hline
 \verb|source/taco_lefse.c| & Nei  \\ 
 \hline
  \verb|source/taco_saus.c| & Nei  \\ 
 \hline
   \verb|Main/*| & Nei  \\ 
 \hline
  \verb|.github/*| & Nei \\
 \hline 
\end{tabular}
\end{center}

\section{Introduksjon - Praktisk rundt øvingen}

 \textit{GNU make} er et automatisk byggeverktøy som kan gjøre store prosjekter lettere å håndtere. Automatisk bygging er spesielt nyttig dersom det tar lang tid å gjenkompilere hver eneste fil hver gang en fil endres. Med et automatisk byggeverktøy, kan man i praksis definere et forhåndsbestemt hierarki av hvilke filer som avhenger av hvilke andre filer. Med et slikt hierarki kan et automatisk byggeverktøy selektivt kompilere kun de filene som har endret seg og filene som er avhengig av de endrede filene.
 
 I praksis har mange programmeringsspråk sine egne byggeverktøy: \verb|gb| for \verb|golang|, \verb|rake| for \verb|ruby|, \verb|mix| for \verb|elixir|, og \verb|rebar| for \verb|erlang|. Fordelen med verktøy som \textit{GNU make} (fra nå av kalt \verb|make|) og \verb|ninja| er at de ikke er bundet til et spesifikt språk. Dette gjør \verb|make| veldig allsidig som fører til at \verb|make| kan bygge hva som helst, så lenge brukeren kan kommandoene som skal kalles og i hvilken rekkefølge.
 
 Introduksjon \ref{sec:2-innforing} gir en innføring i bruk av \verb|make|. Oppgavene finner dere i seksjon \ref{sec:3-oppgave}. I tillegg er det inkludert mer informasjon i appendiks \ref{4-appendiks} som ikke er obligatorisk, men er nyttig for heis-labben for en mer elegant bruk av \verb|make|. For mer informasjon om \verb|make|, ligger manualen (utgitt av Free Software Foundation) ute på \href{https://www.gnu.org/software/make/manual/make.html}{https://www.gnu.org}.
 

\section{Introduksjon - Grunnleggende make}\label{sec:2-innforing}

\verb|make| bygger prosjektet utifra en \verb|Makefile| (\verb|makefile| \ er også godkjent, men har lavere prioritet enn \verb|Makefile|). Denne filen definerer et forhåndsbestemt hierarki og består av et sett med regler - som enten er en målfil (en fil som skal bygges), eller et generelt mål (en oppgave som skal utføres, uavhengig av om sluttproduktet er en fil). Alle regler følger samme mønster:

\begin{lstlisting}
mål : ingredienser
        oppskrift
\end{lstlisting}

\subsection{Generell virkemåte}

På starten av regelen er det definert et sluttprodukt (et mål). For å genere målet, forteller man \verb|make| at en rekke ingredienser, spesifisert etter kolonnen, er nødvendig. Dersom alle disse ingrediensene er å oppdrive, vil \verb|make| følge oppskriften spesifisert under. Om en eller flere ingredienser mangler, vil \verb|make| forsøke å finne mål som kan bygge dem. \textcolor{RWTHrot100}{\textbf{Viktig at det bare er ett eneste tabulatorinnrykk mellom starten av linjen og stegene i oppskriften!}}.

Et eksempel på en regel for å genere filen \verb|main.o|, kan sees her:

\begin{lstlisting}
main.o : main.c constants.h
        gcc -c main.c -o main.o
\end{lstlisting}

Denne snutten leses slik: ``Filen \verb|main.o| avhenger av filene \verb|main.c| og \verb|constants.h|''. For å bygge  \verb|main.o|, kalles \verb|gcc -c main.c -o main.o|. Om hverken \verb|main.c| eller \verb|constants.h| har endret seg siden \verb|make| sist bygde \verb|main.o|, vil ingenting skje.

I motsetning til reelle mål, har man også oppgaver som skal fullføres, men som i seg selv ikke produserer en håndfast fil (veldig vanlig å definere disse reglene med deklarasjonen \verb|.PHONY|):
\\
\begin{verbatim}
.PHONY: clean
clean :
        rm -f *.o
\end{verbatim}


Deklarasjonen \verb|.PHONY| er egentlig ikke nødvendig så lenge man ikke har en fil som er kalt \verb|clean| i prosjektet. Det er uansett anbefalt å bruke \verb|.PHONY| for leselighet. I tillegg til at denne regelen ikke produserer en håndfast fil, så er ikke \verb|clean| avhengig av noen filer, fordi den ikke spesifiserer noe bak kolonen. Akkurat denne kommandoen er spesielt nyttig, ettersom den fjerner alle objektfilene som har blitt kompilert. 

\subsection{Nøstede regler}

Det er vanlig å nøste regler. Et eksempel kan sees under:

\begin{lstlisting}
taco : ost.o lefse.o saus.o protein.o
        gcc -o taco ost.o lefse.o saus.o protein.o

ost.o : ost.c
        gcc -c ost.c

lefse.o : lefse.c
        gcc -c lefse.c

saus.o : saus.c
        gcc -c saus.c

protein.o : protein.c
        gcc -c protein.c
\end{lstlisting}

Dette er en regel for å bygge en tacosimulator \verb|taco|, som avhenger av en rekke filer (\verb|ost.o, lefse.o, saus.o, protein,o|) som må kompileres før selve den kjørbare filen i kan bygges. Hvis ikke alle objektfilene er til stede, vil \verb|make| fortsette å lete nedover for å finne en regel for å bygge det som mangler. I dette tilfellet vil \verb|make| prøve å bygge \verb|taco| først. Dersom \verb|ost.o| ikke finnes vil \verb|make| kompilere \verb|ost.o| ved å lete nedover i koden. Deretter vil \verb|make| fortsette med \verb|taco|-regelen.

I utgangspunktet er det viktig å spesifisere hvilken av reglene som skal kalles. Dersom man kaller \verb|make| fra kommandolinjen, vil \verb|make| finne den første regelen
som ikke starter med et punktum, og så forsøke å utføre den. Alle andre
regler vil bli ansett som hjelperegler for toppmålet.

Det er hovedsakelig to måter å overstyre denne oppførselen på: Først og
fremst kan man manuelt spesifisere hvilken regel \verb|make| skal behandle, ved å
kalle eksempelvis \verb|make tiles.o|. Den andre måten er å spesifisere en variabel kalt \verb|.DEFAULT_GOAL| i makefilen. I følgende eksempel vil \verb|production| bygges, med mindre man eksplisitt ber
om \verb|debug|, selv om \verb|debug| er definert før \verb|production|:

\begin{lstlisting}
.DEFAULT_GOAL := production

debug : main.c
        gcc main.c -O0 -g3

production : main.c
        gcc main.c -O3
\end{lstlisting}

hvor \verb|-O0 -g3| er vanlige flag som blir gitt til \verb|gcc| for å kompilere \verb|main.c|. \verb|-O0| reduserer tiden det tar for å kompilere koden, mens \verb|-g3| brukes for å generere debugging-informasjon. I noen situasjoner kan det være lurt å ikke kompilere \verb|gcc| med \verb|-O0| fordi det fjerner muligheten til å optimalisere ytelsen til koden. 

\subsection{Variabler i regler}

For å ikke ha unødvendig \textit{boilerplate} i reglene, er det vanlig å definere variabler som blir substitutert inn med dollartegnet (\verb|$|). Et eksempel kan sees under:

\begin{lstlisting}
OBJ = ost.o lefse.o saus.o protein.o

taco : $(OBJ)
        gcc -o taco $(OBJ)
\end{lstlisting}

Konvensjonen er å definere variabler med store tegn (kommer fra \textit{shellscripting}), men det er opp til brukeren om denne konvensjonen følges eller ei.

\subsubsection{De to variabel-variantene}

\verb|make| har to forskjellige varianter av variabler - \textit{rekursive} og \textit{enkle}. En \textit{rekursiv} variabel ekspanderer til hva enn variabelen referer til. For eksempel, vil \verb|default| i kodesnutten under, skrive ut variabelen \verb|LEFSE|, som referer til variabelen \verb|MEL|, som igjen referer til variabelen \verb|KORN|, som til slutt inneholder strengen \verb|"karbohydrater"|. I dette tilfellet vil output være strengen \verb|"karbohydrater"| dersom man kaller \verb|make|.

\begin{lstlisting}
LEFSE = $(MEL)
MEL = $(KORN)
KORN = "karbohydrater"

default:
        echo $(LEFSE)
\end{lstlisting}

Problemet med \textit{rekursive}-variabler er at man ikke kan skrive koder som dette:

\begin{lstlisting}
CFLAGS = $(CFLAGS) -O0 -g3
\end{lstlisting}
Denne kodesnutten vil føre til en evig løkke. For å realisere denne type oppførsel kan man bruke \textit{enkle} variabler. Enkle variabler settes med enten \verb|:=| eller \verb|::=|\footnote{\textit{GNU make} støtter begge, og de er ekvivalente, men \textit{POSIX} definerer kun \texttt{::=}}:
\begin{lstlisting}
X := "sjokolade"
Y := "$(X)kake"
X := "gulrotkake"
\end{lstlisting}

Når \verb|make| kommer over denne kodesnutten, finner den verdien av variablene, og bruker så disse verdiene i resten av byggeprosessen. Denne koden er derfor ekvivalent med denne:

\begin{lstlisting}
Y := "sjokoladekake"
X := "gulrotkake"
\end{lstlisting}

\subsubsection{Måter å sette variabler}

I tillegg til at \verb|make| har to forskjellige variabelvarianter, er det mange mulige måter å sette verdiene på disse på. Om man ønsker at en variabel får en verdi, men bare hvis den ikke allerede er definert, bruker man \verb|?=|. For å legge til ledd i en variabel, kan \verb|+=| brukes. For å kjøre et shellscript og tilegne resultatet til en variabel, brukes \verb|!=|. \verb|make| har i tillegg støtte for avdefinerisering av en variabel med deklarasjonen \verb|undefine|. \verb|make| kan også definere multilinjevariabler slik:

\begin{lstlisting}
define LINES =
"Linje en"
$(LINJE_TO)
endef
\end{lstlisting}

\subsubsection{Spesielt lange variabellister}

Dersom man har spesielt lange variabellister, er det mulig å bruke \verb|\| for å signalisere at linjen ikke ender selv ved et linjeskift. Sett nå at \verb|taco| bestod av flere filer enn \verb|ost|, \verb|lefse|, \verb|saus|, og \verb|protein|. Man kan da definere \verb|OJB| som:

\begin{lstlisting}
OBJ = ost.o lefse.o saus.o protein.o tacokrydder.o\
        avokado.o rømme.o bønner.o nachos.o mais.o\
        paprika.o løk.o olje.o
\end{lstlisting}

\subsection{Infererte regler}

\textit{GNU make} har en stor fordel når det kommer til kode som er skrevet med \verb|C| eller \verb|C++|. \verb|make| vil anta at en fil kalt \verb|kardemommete.o| avhenger av en tilsvarende \verb|kardemommete.c|-fil. Videre vil \verb|make| anta at kompilatorflagget \verb|-c| \ brukes for å genere objektfiler. 

Dette kan brukes for å simplifisere \verb|taco|-eksempelet. Det eneste som gjenstår er å fortelle \verb|make| hvilken kompilator som brukes. Dette vil \verb|make| klare å tyde ut fra hvilken kommando som lenker sammen objektfilene - automatisk. \verb|taco|-eksemplet kan dermed simplifiseres til:



\begin{lstlisting}
OBJ = ost.o lefse.o saus.o protein.o

taco : $(OBJ)
        gcc -o taco $(OBJ)

ost.o :
lefse.o :
saus.o :
protein.o :
\end{lstlisting}

Siden objektfilene ikke avhenger av noe annet enn de korresponderende \verb|.c|-filene, er det nok å skrive:

\begin{lstlisting}
OBJ = ost.o lefse.o saus.o protein.o

taco : $(OBJ)
        gcc -o taco $(OBJ)
\end{lstlisting}


Om alle objektfilene viser seg å avhenge av verdiene som er definert i \texttt{råvare\_pris.h}  kan dette beskrives enkelt slik:

\begin{lstlisting}
OBJ = ost.o lefse.o saus.o protein.o

taco : $(OBJ)
        gcc -o taco $(OBJ)
        
$(OBJ) : råvare_pris.h
\end{lstlisting}


Det er diskutabelt om denne måten å lage \verb|make|-filer er å foretrekke, siden
det ikke lenger er helt klart hva som skjer - men til syvende og sist er det
rett og slett et spørsmål om personlig smak.

\subsection{Betingelser i regler}

Akkurat som i vanlige programmeringsspråk, kan man bruke betingelser for å teste en betingelse før en handling/regel blir utført. Betingelser kan være veldig nyttige, spesielt om man har et prosjekt som skal kunne bygges på forskjellige plattformer\footnote{Om man først skal støtte flere plattformer, er nok verktøyet \texttt{cmake} verdt å ta en titt på.}. Et eksempel på dette kan sees under: 

\begin{verbatim}
GCC_LIBS = -lgnu
DEFAULT_LIBS = -lsystem_specific

ifeq ($(CC), gcc)
       $(CC) -o prog $(OBJ) $(GCC_LIBS)
else
       $(CC) -o prog $(OBJ) $(DEFAULT_LIBS)
endif
\end{verbatim}


hvor \verb|ifeq| betyr \textbf{if eq}ual. Det er ikke nødvendig med en \verb|else| for å bruke \verb|ifeq| - og en \verb|else if| bruker syntaksen for \verb|else| med \verb|ifeq| foran: 

\begin{lstlisting}
ifeq ($(CC), gcc)
        LIBS = $(GCC_LIBS)
else ifeq ($(CC), clang)
        LIBS = $(CLANG_LIBS)
else
        LIBS = $(DEFAULT_LIBS)
endif
\end{lstlisting}

I tillegg støtter \textit{GNU make} også andre tester enn \verb|ifeq|: eksempelvis \verb|ifneq| for test av ulikhet, \verb|ifdef| for å teste om noe er definert, eller \verb|ifndef| for å teste om noe ikke er definert. For \verb|ifdef| og \verb|ifndef| tar operatoren kun ett argument, ikke to:

\begin{lstlisting}
ifdef $(USE_SYSTEM_LIBS)
        LIBS += -lsystem_specific
endif
\end{lstlisting}

%\subsection{Tillegsi}
\end{alphasection}              


\setcounter{section}{0}